% !TEX root = ../main.tex

\chapter{Cluster Analysis}

% 这是中文
% 在原始的数据中,我们将艺术家划分了流派,但是一个人创作的音乐可能有着不同的风格,因此我们要对音乐的数据进行聚类分析,并对其进行分类。在聚类的方法上,选用了K-means聚类和层次聚类两种方法,并对聚类的结果进行了评估。


\section{Data Preprocessing}
% 在进行聚类分析之前,需要对数据进行预处理。将数据处理成多维度的向量,可以将每个向量视为一个样本点,从而计算样本点间的距离,
\subsection{Entropy Weight Process}
% 首先在数据的选择上,由于原始数据有将近十万首歌曲的数据,我们只需要对有一定影响力的艺术家创作的歌曲进行分析,因此选择由熵权法确定的影响力前100位的艺术家创作的歌曲,共计2万多首。 
% 在具体的数据处理上,前文第二章中已经能够通过熵权法确定影响力前100位的艺术家名单,使用pandas对导出的表格进行选取,提取出对应艺术家的所有创作的歌曲。

\subsection{PCA Dimensionality Reduction}
% 音乐的特征比较繁杂,不易提取特征,所以聚类的时候使用的是使用PCA将数据维度从12维降低到7维的数据,对结果进行处理。聚类的时候也加上了PCA里面的权重
% PCA的权重部分的处理在第一章中已完成,我们知道了各个特征的PC值与权重,将数据从12维重新映射到7维上
% tabel1: PCA.csv 
% table2: original music data 
% table3: final music data

% https://blog.csdn.net/qq_27586341/article/details/103909954
% https://www.jianshu.com/p/794e91f60170
% https://www.latexstudio.net/archives/51640.html

\section{Cluster Method}
% 主要使用K-means和层次聚类的方法



\subsection{K-means Cluster}
% 聚类方法介绍

\subsection{Hierarchical Cluster}
\subsubsection{原理}
% 层次聚类的方法是先计算样本间的距离,每次将距离最近的点合并到同一个类。然后,再计算类与类之间的距离,将距离最近的类合并为一个大类。不停的合并,直到合成了一个类。其中类与类的距离的计算方法有:最短距离法,最长距离法,中间距离法,类平均法等。比如最短距离法,将类与类的距离定义为类与类之间样本的最短距离。
% 层次聚类算法根据层次分解的顺序分为:自下向上和自上向下,即凝聚的层次聚类算法和分裂的层次聚类算法(agglomerative和divisive),也可以理解为自下而上法(bottom-up)和自上而下法(top-down)。
% 自下而上法:凝聚型层次聚类,就是一开始每个个体(object)都是一个类,然后根据linkage寻找同类,最后形成一个“类”。
\subsubsection{流程}
\subsubsection{优缺点}

\section{Cluster Evaluation}

\subsection{Within-cluster Sum of Squared Errors(SSE)}
% 簇内误差平方和

\subsection{Silhouette Analysis}
% 轮廓系数silhouette coefficient

\section{Visualization of Clusters}