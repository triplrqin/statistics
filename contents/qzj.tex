% !TEX root = ../main.tex

\chapter{Cluster Analysis}

% 这是中文
% 在原始的数据中,我们将艺术家划分了流派,但是一个人创作的音乐可能有着不同的风格,因此我们要对音乐的数据进行聚类分析,并对其进行分类。在聚类的方法上,选用了K-means聚类和层次聚类两种方法,并对聚类的结果进行了评估。


\section{Data Preprocessing}
% 在进行聚类分析之前,需要对数据进行预处理。
\subsection{Entropy Weight Process}
%首先在数据的选择上,由于原始数据有将近十万首歌曲的数据,我们只需要对有一定影响力的艺术家创作的歌曲进行分析,因此选择由熵权法确定的影响力前100位的艺术家创作的歌曲,共计2万多首。 

\subsection{PCA Dimensionality Reduction}
\section{Cluster Method}
% 主要使用K-means和层次聚类的方法

\subsection{K-means Cluster}
% 聚类方法介绍

\subsection{Hierarchical Cluster}

\section{Cluster Evaluation}

\subsection{Within-cluster Sum of Squared Errors(SSE)}
% 簇内误差平方和

\subsection{Silhouette Analysis}
% 轮廓系数silhouette coefficient

\section{Visualization of Clusters}